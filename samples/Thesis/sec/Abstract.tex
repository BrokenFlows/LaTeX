% ABSTRACT
\begin{abstract}
	\thispagestyle{plain}
	\setcounter{page}{4}
	\phantomsection
	\addcontentsline{toc}{chapter}{Abstract}
	\ac{noma} will be an important technology in enabling \ac{fra} and the next generation of wireless networks.
	\ac{noma} shares orthogonal resources amongst connected devices to improve spectral efficiency.
	This is done through power domain multiplexing within those resources.

	\par
	This thesis will examine the success of \ac{noma} when interfacing \ac{bf} and \ac{mmwave} spectrum, which are both to be facets of 5G networks as well.

	\par
	A review of the literature pertaining to \ac{noma} and related technologies is conducted.
	Challenges facing \ac{noma}, such as receiver complexity and \ac{plr} are outlined and later tackled.

	\par
	A realistic \ac{mmwave} channel model is chosen as the basis for all simulations of \ac{noma} systems.
	This benefits the work as it puts it in the correct frame of reference.

	\par
	The performance of \ac{zf} and \ac{mrt} \ac{bf} algorithms are analysed and \ac{mrt} is chosen as the more successful approach for \ac{noma}.
	
	\par
	Different schemes are considered and optimal \ac{pa} is achieved with significantly reduced processing.
	\ac{hnoma} with \ac{pf} is found to be a scheme which realises the best sumrates, other than full \ac{noma}, while reducing the complexity and overhead of full \ac{noma}.
\end{abstract}


% LAY-ABSTRACT
{\renewcommand*\abstractname{Lay-Abstract}
\begin{abstract}
	\thispagestyle{plain}
	\setcounter{page}{5}
	\phantomsection
	\addcontentsline{toc}{chapter}{Lay-Abstract}
	This thesis explores \acf{noma} and how it can be used with \acf{bf} and \acf{mmwave} frequencies.
	
	\par
	Cellular networks communicate with devices through radio waves, which includes \ac{mmwave} frequencies.
	\ac{bf} is a way for cell towers to steer signals towards those devices.

	\par
	Traditionally, cell towers may have communicated with one device at a time, or using a single resource per device.
	\ac{noma} allows multiple devices to share those same resources.
	This is important due to the finite nature of those resources.

	\par
	This thesis proposes multiple methods of combining these technologies in future mobile networks and finds that not only is it feasible but that sharing resources, as \ac{noma} aims to, is beneficial.
	
\end{abstract}}

