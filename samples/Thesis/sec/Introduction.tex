\chapter{Introduction}
\ac{noma} is an important \ac{ma} technology for the future of wireless networks and has great potential for 5G.
With a trend towards \ac{mmwave}, MIMO, and \ac{bf} it is important to consider the strengths and weaknesses of these technologies together and how they can be used in future radio networks.

\par
\ac{noma} is different from conventional \ac{oma} technologies by allowing more than one \ac{ue} to access shared resources at a time.
This is ever more important as demands for connectivity rise and increase the importance of spectral efficiency in our communications.

\par
The challenges faced by \ac{noma} will be discussed and potential solutions will be evaluated.
Analysis will be done through the lens of \ac{mmwave} spectrum and with the aim of maximising the sumrates of systems, which use \ac{noma} for \ac{ma} and have a \ac{bs} which can use \ac{bf}.

\par
\ac{mmwave} spectrum will bring greater bandwidths and has the potential for much higher capacity.
However, it brings its own set of limitations as well.
Higher frequencies will attenuate signals further and \ac{mmwave} will have further challenges still as it is less likely to refract around, or penetrate through, obstacles as other \ac{rf} signals do..

\par
This thesis will consider these challenges and propose solutions through various \ac{noma} schemes and means of implementing \ac{noma} which reduce adverse effects.
First, a review of the current literature on these topics is conducted.
Then, the technologies are tested by series of simulations, comparing and contrasting various aspects of \ac{noma} systems.

\iffalse
This thesis will evaluate the current state of \ac{noma} and related technologies.
We will also explore the work undertaken in and related to \ac{noma} simulations to date.
Finally, we will consider the work required to complete this project.
\fi

