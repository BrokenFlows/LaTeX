\chapter{Background/ Literature Review}
In this chapter we will explore existing \ac{ma} techniques, basic \ac{noma} systems, and research to date within the field of \ac{noma}.

% Background on Current MA
\section{Existing Multiple Access Technologies}

\ac{ma} techniques allows multiple devices to share a medium in communications.
These techniques improve the efficiency of communicating over a medium and are relevant to mobile communications networks, as they use finite media to communicate wirelessly with mobile \acp{ue}.

\par
\ac{ma} techniques used, in mobile networks to date, include:

\begin{itemize}
	\item[\textbf{1G}] \ac{fdma}
	\item[\textbf{2G}] \ac{tdma}
	\item[\textbf{3G}] \ac{cdma}
	\item[\textbf{4G}] \ac{ofdma}
\end{itemize}

\par
This is due to the constant growth in mobile use and data use in particular.

\par
While \ac{fdma} and \ac{tdma} allow users to use one frequency or time resource, respectively, newer \ac{ma} allows improved sharing of resources.
\ac{cdma} allowed multiple \acp{ue} to use the same frequencies simultaneously, separating signals with orthogonal codes.
\ac{ofdma} or \ac{oma} uses orthogonal frequencies to allow more \acp{ue} to communicate contemporaneously within a tighter grouping of frequencies.

\par
Though \ac{cdma} and \ac{ofdma} have improved spectral efficiency compared to \ac{fdma} and\ac{tdma} sharing orthogonal resources, as in \ac{noma}, increases the spectral efficiency of the system further still.
Consider the throughput of a \ac{ceu} in an \ac{oma} system.
In \ac{oma} the \ac{ceu} claims the whole resource but the throughput of the device is low due to channel conditions at the cell edge.
Under \ac{noma}, \ac{ceu} can share those resources with \ac{ccu} that have better channel conditions, and therefore more efficient use of the spectrum.

% Background on NOMA
\section{Basic NOMA}
Power-Domain \ac{noma} can serve multiple users in the same orthogonal resource
(time slot, orthogonal code, or orthogonal frequency)
by using power multiplexing to perform multiple access.
This is done by allocating different powers to different users.
\cite{ding:2017}
Assigning different power levels to individual users allows each user to decode their signal using Interference Cancellation techniques.

\par
\Fref{fig:basicnoma} shows a simple two-user and \ac{bs} \ac{noma} downlink system.
The following description of a Basic \ac{noma} system follows \cite{saito:2013,shahab:2018,ding:2017}.
Taking \ac{ue}\tsb{1} as a \ac{ccu} and \ac{ue}\tsb{2} as the \ac{ceu}, we can say that their distances $d_i$ from the \ac{bs} fit $d_1 \leq d_2$.
For the same carrier frequency, we can then say that the channel gains, $\abs{h_i}^2$, at the \ac{ue}'s fit
$\abs{h_1}^2 \geq \abs{h_2}^2$.
The powers, $P_i$, can then be described by $P_1 \leq P_2$, where $P_1 + P_2 = P_{total}$.
The signals sent to the \ac{ue}'s are superposed atop each other, summing to the maximum power of the system.

\begin{figure}[htb]
	\centering
	\includegraphics[width=1\linewidth]{../images/NOMA_SIC_Example_dark.png}
	\caption{Basic NOMA with SIC for Two UE Receivers in Downlink}
	\label{fig:basicnoma}
\end{figure}

\par
To decode the signals, in downlink \ac{noma}, \ac{sic} is used.
For $N$ users with channel gains $\abs{h_i}^2$ if
$\abs{h_1}^2 \geq \abs{h_2}^2 \geq \ldots \geq \abs{h_N}^2$
then each user applies \ac{sic} to
$N-i$ signals before decoding their own.
In the two user case that means that \ac{ue}\tsb{2} only decodes their own message, treating \ac{ue}\tsb{1}'s message as noise, while \ac{ue}\tsb{1} performs \ac{sic} then decodes its message.

\par
The achievable rate is given generally by \fref{eq:nomagen}:

\begin{equation}
	R_i = \log_2 
	\left(
	1 +
	\frac{P_i \cdot \left| h_i \right| ^2}
	{\sum_{n=1}^{i-1}P_{n} \cdot \left| h_i \right| ^2 + N_0}
	\right)
	\label{eq:nomagen}
\end{equation}

\par
In the specific case of a 2-\ac{ue} \ac{noma} system the rates of \ac{ue}\tsb{1} and \ac{ue}\tsb{2} are given by \fref{eq:noma1} and \fref{eq:noma2} respectively:

\begin{equation}
	R_1 = \log_2 
	\left(
	1 +
	\frac{P_1 \cdot \left| h_1 \right| ^2}
	{N_0}
	\right)
	\label{eq:noma1}
\end{equation}

\begin{equation}
	R_2 = \log_2 
	\left(
	1 +
	\frac{P_2 \cdot \left| h_2 \right| ^2}
	{P_{1} \cdot \left| h_2 \right| ^2 + N_0}
	\right)
	\label{eq:noma2}
\end{equation}

\par
As was discussed above, \fref{eq:noma1} shows that \ac{ue}\tsb{1} suffers no interference from \ac{ue}\tsb{2}.
Though \ac{ue}\tsb{1} is assigned less power, which hinders its rate, being able to remove the interfering signal boosts its rate.
The converse is true for \ac{ue}\tsb{2} which is assigned more power but has its rate hindered by the interfering signal of \ac{ue}\tsb{1}, shown in \fref{eq:noma2}.

% Beamforming Algorithms
\section{Beamforming Algorithms}
\ac{bf} is a process by which devices with multiple antennas can control the direction of their transmission.
A transmission using \ac{bf} has two components:

\begin{itemize}
	\item \acl{pa}
	\item Direction
\end{itemize}

\par
Direction is given by what is called a \ac{bf} precoder.
The purpose of a \ac{bf} algorithm is to determine a precoder for a \ac{ue}.
This section will consider two popular \ac{bf} algorithms.

\subsection{Zero Forcing Beamforming}
This subsection follows details of \ac{zf} \ac{bf} outlined in \cite{zf,zf2}.
The primary aim of \ac{zf} \ac{bf} is to nullify inter-\ac{ue} interference by forcing the power transmitted in the direction of other \acp{ue} to zero, hence the name.
\ac{zf} focuses on the nullification of interference over maximising power in the direction of the target \ac{ue}.
Direction of a \ac{ue} is determined by the channel, assumed through \ac{csi}.

\par
The nulling property of \ac{zf} \ac{bf} is what makes it of interest for \ac{noma} systems.
As \acp{ceu} must endure inter-\ac{ue} interference, as they do not perform \ac{sic}, \ac{zf} may reduce the interference they experience, increase their throughput, and by extension increase the sumrate of the system.
This increase in \ac{ceu} throughputs may be sufficient to overcome the fact that \ac{zf} \ac{bf} does not maximise power in the direction of the target \ac{ue}.

\par
\ac{zf} may cause issues for \ac{noma} as the inter-\ac{ue} interference is what \acp{ccu} use to do \ac{sic} and later reduce their interference.
The effects of this may be seen in the results of future simulations.

\par
\Fref{eq:zf} shows the \ac{zf} algorithm used to generate the precoder.

\subsection{Maximum Ratio Transmission}
This subsection follows details of \ac{mrt} \ac{bf} outlined in \cite{mrt,mrt2}.
\ac{mrt} contrasts with \ac{zf} as this algorithm prioritises the maximisation of each individual \ac{ue}'s throughput over all else.
By prioritising the throughput of the target \ac{ue}, other \acp{ue} may experience some degree of interference by neglecting to consider their presence.

\par
\ac{mrt} assumes that the benefits of maximising transmission power outweigh the cons.
In a \ac{noma} system this means that sumrate may increase as the signal strength of each \ac{ue} is maximised.
The sumrate of each \ac{ue} is also impacted, however, by all other \acp{ue} having their signal strength maximised.

\par
As \ac{mrt} does cause inter-\ac{ue} interference there should be no hindrances to the \ac{sic} process.
This leaves \ac{mrt} indifferent when considering \acp{ccu} but may make it a worse option for \acp{ceu}.

\par
\Fref{eq:mrt} shows the \ac{mrt} algorithm used to generate the precoder.

% Optimising NOMA
\section{Optimising NOMA}
There have been many investigations into how to improve upon basic \ac{noma} techniques.
In this section we will explore some of these improvements.

\subsection{Hybrid NOMA}
\label{sec:hnoma}
\ac{hnoma} is the concept of creating a \ac{noma} system within an orthogonal resource of another \ac{oma} system.
For example, within an \ac{ofdma} resource two \acp{ue} could share the orthogonal frequency and treat it as a single \ac{noma} resource \cite{ding:2017}.
This is useful as to serve all \acp{ue} in a cell on a single \ac{noma} resource would not be feasible.

\par
The idea is highly dependant on \ac{uppa} as a greater \ac{plr} will increase the rates achievable by all \acp{ue} involved \cite{ding:2016}.
\ac{hnoma} has appeal in the literature as existing \ac{oma} techniques are mature and well understood.
The blending of traditional techniques with a means of improving spectral efficiency is very appealing.

\par
It is noteworthy that a \ac{ccu} may achieve a better rate with \ac{noma} than with traditional \ac{oma} while a \ac{ceu} is likely to achieve a worse rate with \ac{noma} than \ac{oma} \cite{ding:2017}.

\subsection{Grouping UEs}

\par
Related to \ac{hnoma}, but also an area of interest in basic \ac{noma} too, is the pairing of \acp{ue} in \ac{noma} systems.
\ac{uppa} is crucial to the success of \ac{noma} which has lead to many investigations into \ac{hnoma} and \ac{pa} algorithms.

\par
One such investigation was into a Tree-search based Transmission Power Assignment \cite{li:2014}.
This is particularly novel for it is an ideal solution without conducting an exhaustive search.

\subsection{mmWave NOMA}
The use of \ac{mmwave} spectrum as a carrier for \ac{noma} and \acl{bf} is well known.
Limited spectrum in our current frequency bands makes \ac{mmwave} spectrum the obvious home for future growth.

\par
It has been noted that \ac{mmwave} frequencies are well aligned for both \acl{bf} and \ac{noma} as they both benefit from MIMO \cite{ding:2017}.
This is as a result of \ac{mmwave} allowing for smaller antennas, which in turn allow for tighter antenna arrays.
MIMO is of course essential for \acl{bf} and benefits \ac{noma} as strong channel vector  correlation is beneficial to both MIMO and \ac{noma}.

% Limitations and Future Challenges - Potential Project Directions
\section{Limitations \& Challenges}

The following section addresses some of the main limitations and future challenges against \ac{noma}.
These are areas which may require research and are potential directions for this project to work.

\subsection{Path Loss Ratio}
\ac{noma} can may only be applied when the \ac{plr} between \ac{ue}'s is roughly \SI{8}{\decibel} or higher and the rates of \ac{noma} users improves with greater \acp{plr}\cite{ding:2016,yazaki:2014,islam:2017web}.
This poses as a limitation in how users can be paired in \ac{noma} and how power can be allocated at the \ac{bs}.

\par
Due to the critical importance of \ac{pa} and \acp{plr} to the success of \ac{noma} coming up with effective \ac{pa} algorithms is of the utmost importance.
Similarly, for \ac{hnoma}, \ac{uppa} is of great importance as there is the potential to split \acp{ue} into different \ac{oma} resources as well as performing \ac{pa}.

\subsection{Receiver Complexity}
The \ac{sic} aspect to \ac{noma} introduces a lot of complexity to the receivers \cite{islam:2017,islam:2017web}.
The \ac{sic} order, needs to be communicated to the receivers.
\ac{sic} needs to be carried out correctly or errors will propagate and require retransmission, lowering the throughput of the system.
There is also additional overhead simply in the decoding of the signal.
All of these factors may hinder make the receivers harder to design and more energy demanding to run.

\subsection{Quantisation Error}
Quantisation error in the \ac{adc} may cause errors for the receiving \ac{ue} during the signal processing \cite{islam:2017}.
As some signals will be very weak, particularly compared to the large signals that are to be removed by \ac{sic}, the \acp{adc} must support a large input range with a very high resolution.
High resolution \acp{adc} can be cost prohibitive and but a certain threshold must be met to support \ac{noma}.
A trade off will have to be made between quantisation error and the performance of \ac{sic} in these \ac{noma} receivers.

