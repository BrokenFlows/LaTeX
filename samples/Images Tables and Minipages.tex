\documentclass[A4]{article}

\usepackage{stdcd}

\title{COMP40660 Assignment 3}
\author{Cian Dowd}
\date{\today}

\begin{document}
\maketitle

\section*{Part 1}

\begin{figure}[H]
	\centering
	\includegraphics[width=\linewidth]{nyquist}
	\caption{Nyquist Capacity versus Bandwidth}
	\label{fig:nyquist}
\end{figure}

\section*{Part 2}

\begin{figure}[H]
	\centering
	\includegraphics[width=\linewidth]{shannon}
	\caption{Shannon Capacity versus SNR (\si{\decibel})}
	\label{fig:shannon}
\end{figure}

\section*{Part 3}

\newcolumntype{R}[1]{>{\raggedleft\arraybackslash}m{#1}}

\begin{table}[H]
	\centering
	\begin{minipage}{0.45\linewidth}
		\centering
		\caption{Suitable Modulation Formats}
		\label{tab:M}
		\begin{tabular}{ r | R{1.0cm}  R{1.0cm}  R{1.0cm}  R{1.0cm} }
			& \SI{20}{\mega\hertz} & \SI{40}{\mega\hertz} & \SI{80}{\mega\hertz} & \SI{160}{\mega\hertz}\\
			\hline
			\SI{5}{\decibel} & 2 & 2 & 2 & 2\\
			\SI{15}{\decibel} & 4 & 4 & 4 & 4\\
			\SI{20}{\decibel} & 8 & 8 & 8 & 8\\
			\SI{30}{\decibel} & 16 & 16 & 16 & 16\\
		\end{tabular}
	\end{minipage}
	\begin{minipage}{0.45\linewidth}
		\centering
		\caption{Suitable Modulation Formats}
		\label{tab:M}
		\begin{tabular}{ r | R{1.0cm}  R{1.0cm}  R{1.0cm}  R{1.0cm} }
			& \SI{20}{\mega\hertz} & \SI{40}{\mega\hertz} & \SI{80}{\mega\hertz} & \SI{160}{\mega\hertz}\\
			\hline
			\SI{5}{\decibel} & BPSK & BPSK & BPSK & BPSK\\
			\SI{15}{\decibel} & QPSK & QPSK & QPSK & QPSK\\
			\SI{20}{\decibel} & 8PSK & 8PSK & 8PSK & 8PSK\\
			\SI{30}{\decibel} & 16QAM & 16QAM & 16QAM & 16QAM\\
		\end{tabular}
	\end{minipage}
\end{table}

\section*{Part 4}

\begin{itemize}
	\item
		Different models will have different relations to real wireless scenarios, though theoretical capacity values are typically overly optimistic.
		This is a result of theoretical capacity values failing to account for real scenarios with their assumptions.
		Some of the differences between the two models we are considering and real wireless scenarios are:
		\begin{enumerate}
			\item[\textbf{Nyquist}]
				assumes zero noise.
				This is inaccurate as there will always be at least some background noise and at times there will be other interferers.
			\item[\textbf{Shannon}]
				assumes only additive white Gaussian Noise.
				This is a more realistic assumption than that of Nyquist's formula as there will always be some noise.
				Shannon's capacity formula becomes unrealistic in the presence of more complex noise and interference.
		\end{enumerate}

		SNR can be affected by many factors, directly affecting either the signal strength or the strength of the noise.\\

		Factors attenuating signal strength:

		\begin{itemize}
			\item
				Free Space Loss
			\item
				Feeder Lines
			\item
				Filters
		\end{itemize}
		
		Factors increasing noise:
		
		\begin{itemize}
			\item
				Background radiation in the channel
			\item
				Interferers
			\item
				Thermal noise in hardware
		\end{itemize}

	\item
		The modulation limits affect the SNR, and therefore the likelihood of receiving the correct symbol, and the potential throughput of the system.\\
		For high SNR, where the receiver can make a correct decision, the higher modulation schemes allow for better throughput.\\
		For lower SNR, where the receiver may fail to make a correct decision, the higher modulation schemes will result in constellations which are more error prone, so lower modulation schemes will be used as a fallback.
	\item
		The percentage capacity improvement of \texttt{1024QAM} over \texttt{256QAM} is 25\%.
\end{itemize}

\newpage
\section*{MATLAB Code}
\lstinputlisting[final,style=Matlab-editor]{../CSAssignment3.m}

\end{document}
